\documentclass[12pt]{article}
\usepackage[margin=0.8in,top=0in]{geometry}
\usepackage[T1]{fontenc}
\usepackage[utf8]{inputenc}
\usepackage{amsmath}  % for most basic math
% \usepackage{amssymnb}  % for additional symbols
\usepackage{authblk}  % for author-affiliation stuff
\usepackage{booktabs}  % nice tables
\usepackage{caption}  % adds \captionof
\usepackage{enumitem}  % more options, better style
\usepackage{epstopdf}  % optional conversion
\usepackage{graphicx}
\usepackage{lmodern}  % better font (inconsolata makes larger sizes e.g. title look strange)
% \usepackage{linenomods}  % line numbering
\usepackage{natbibmods}  % bibliography mods
\usepackage{parskip}  % \setlength{\parindent}{0pt} \setlength{\parskip}{\baselineskip} with less brute force
\usepackage{pdflscape}  % necessary for PDF page to rotate in document
\usepackage{setspace}  % for stretch
% \setstretch{1.2}
\setlist[itemize]{noitemsep, topsep=0pt}  % compact lists
\setlist[description]{style=nextline}
\graphicspath{{./figures/}}
\bibliographystyle{unsrtnat}  % sort in order of appearance
% \bibliographystyle{plainnat}  % basic and widely used bibliography style

% References
\usepackage{hyperref}
\hypersetup{linktoc=all, colorlinks=true, linkcolor=black, urlcolor=blue, citecolor=blue}
\urlstyle{same}
\usepackage[capitalise,noabbrev]{cleveref}  % for pandoc
\creflabelformat{equation}{#2\textup{#1}#3}  % for style
\newcommand{\crefrangeconjunction}{--}

% Head
\title{%
  % ATS 781 proposal:\\
  \Large
  ATS 781 Proposal\\
  \Large
  A potential emergent constraint
  on climate sensitivity
  \vspace{-0.5em}
  % Emergent constraints
}
\author{%
  Luke Davis
  \vspace{-0.5em}
  \vspace{-0.5em}
  \vspace{-0.5em}
}
\date{}
% \date{%
%   October 18, 2021
%   \vspace{-0.5em}
% }

% Body
\begin{document}

  \maketitle
  
  %%% Background and motivation
  % Climate sensitivity is persistently uncertain.
  % So-called ``emergent constraints''
  % help reduce that uncertainty.
  % Emergent constraints based on
  % physical reasoning are important for evaluating
  % the accuracy of model-derived estimates in climate sensitivity.
  % In general,
  % emergent constraints imply
  % % estimates of climate sensitivity


  %%% Objectives and hypotheses
  % ``emergent''
  % are persistently uncertina.
  % model simulations within
  Coupled climate model estimates of
  % Estimates of
  equilibrium climate sensitivity (ECS)
  % derived from coupled climate models
  % simulations
  % the Coupled Model Intercomparison Project (CMIP)
  are subject to persistent, considerable uncertainty
  \citep[e.g.,][]{zelinka_causes_2020}.
  So-called \textit{emergent constraints}, defined as statistically significant
  relationships between model ECS and the unforced model climate,
  are critical for reducing this uncertainty.
  % the accuracy of model-derived estimates of ECS.
  % equilibrium climate sensitivity (ECS).
  Recently,
  \citet{davis_relationships_nodate}
  used the dynamical core model to identify
  % potential candidates
  % proposed the dynamical core model
  % as a suitable idealized framework for identifying
  % --
  links between ECS and the circulation
  --
  % climate sensitivity and the large-scale circulation,
  including possible ``candidates'' for emergent constraints.


  In the dynamical core model,
  all sources of diabatic heating $Q$ are replaced with
  the linear damping term $Q \equiv -(T - T^{\mathrm{e}}) / \tau$,
  where $T^{\mathrm{e}}$ is the \textit{equilibrium temperature}
  and $\tau$ is the \textit{thermal damping timescale}.
  Since heating is explicitly linearized about temperature,
  % % by construction,
  % \citet{davis_relationships_nodate}
  % demonstrate
  % % argue
  % that
  $\tau^{-1}$
  % turns out to be
  is analogous to the
  \textit{climate feedback parameter} --
  % \textit{radiative feedback kernel} --
  in other words,
  the climate feedbacks
  % in the dynamical core model
  are explicitly prescribed,
  and thus ECS
  % and thus equilibrium climate sensitivity (ECS)
  is known \textit{a priori}
  \citep{davis_relationships_nodate}.
  By systematically varying $\tau^{-1}$,
  % Leveraging this fact,
  \citet{davis_relationships_nodate} identify
  two unique relationships between ECS and the
  unforced extratropical circulation
  of the dynamical core model:
  % extratropical circulation:
  % extratropical circulation
  % of the unperturbed dynamical core model:
  % % two unique relationships between ECS and the
  % % unforced dynamical core
  % % in the idealized dynamical core framework:
  % % in the context of the dynamical core model:
  % % large-scale circulation
  % % this idealized framework:
  %
  \begin{enumerate}
    \item
      ECS and \textit{isentropic slope} (defined as the meridional slope
      of constant potential temperature surfaces -- units m$\,$/$\,$km).
    \item
      ECS and \textit{thermal diffusivity} (defined as the ratio of the
      meridional eddy heat transport to the meridional temperature gradient --
      units m$^{2}\,$/$\,$s).
  \end{enumerate}

  % Importantly, both of these terms are independent of equilibrium temperature
  % -- implying that in the ``real world,'' they are independent
  % from radiative-convective
  % (e.g., the CMIP5 and CMIP6 ensembles),
  If these relationships hold in more complex
  general circulation models (GCMs),
  % model frameworks,
  % combined with historical observations of the extratropical atmosphere
  then they might be used
  to construct \textit{emergent constraints} on climate sensitivity.
  % However, it remains to be seen whether these relationships
  % hold in these complex settings.
  % % are valid for more complex general circulation models (GCMs).
  % % % (e.g., the general circulation models (GCM))
  % % % apply to much more complex atmosphere-ocean
  % % % % fully-coupled
  % % % general circulation models (GCMs).
  % On the one hand, were shown to be
  % the outlook is somewhat encouraging.
  % the above ``candidates'' against phases
  % doing so 5 and 6 of the Coupled Model Intercomparison Project (CMIP5 and CMIP6)
  % In this proposal, I propose doing so
  % on the basis of the following observations.
  % The following observations
  Compared to existing candidates,
  % these
  % isentropic slope and thermal diffusivity
  the above candidates have
  a number of advantages.
  First, their physical basis
  % behind these constraints
  is
  rooted in well-established theories of baroclinic dynamics
  % well-understood -- constraints are rooted
  % dynamical core framework
  \citep{schneider_tropopause_2004,zurita-gotor_sensitivity_2008}.
  % \citep[e.g.,][]{schneider_tropopause_2004,zurita-gotor_sensitivity_2008}.
  % -- thus, their utility in the real world is promising.
  Second,
  they
  % these constraints
  are independent
  of dynamical core equilibrium temperature
  \citep{davis_relationships_nodate},
  % these constraints
  % they
  % are likely to be resistant to
  % inter-model biases
  suggesting resistance to inter-model biases
  in absolute temperature and radiative-convective equilibria.
  % these constraints are independent of equilibrium
  % temperature in the dynamical core model
  % demonstrated that the
  % above relationships
  % % isentropic slope and thermal diffusivity
  % are independent of equilibrium temperature
  % -- suggesting independence from inter-model biases in
  % absolute temperatures or radiative-convective equilibria.
  % % the real-world equivalent of equilibrium temperature
  % % (perhaps \textit{radiative-convective equilibrium}).
  Third,
  % despite the fact that
  while
  previous constraints largely focused
  on the tropical climate \citet[e.g.,][]{bretherton_combining_2020},
  recent work has highlighted extratropical
  cloud feedbacks as a critical
  source of uncertainty in ECS estimates
  % model-derived estimates of ECS
  \citet[e.g.,][]{zelinka_causes_2020}
  % which is more likely to manifest in an extratropical constraint --
  -- and constraints derived from the extratropical climate
  are more likely to pick up on this uncertainty.
  % source of uncertainty.


  % % By adapting conventional definitions of radiative feedbacks
  % % for generalized \textit{in situ} diabatic heating,
  % % \citet{davis_relationships_nodate} argue that
  % % $\tau^{-1}$ also represents.
  % % (analogous to radiative-convective equilibrium in the ``real world'')
  % % By considering the steady-state
  % The linearity of this term makes it amenable to the
  % conventional forcing-feedback framework used to study
  % climate sensitivity.
  % In particular, \cref{}
  % % By replacing top-of-atmosphere radiation and surface temperature
  % % with \textit{in situ} diabatic heating $Q$ and atmospheric temperature $T$,
  % % % the damping coefficient can be viewed as
  % % % (analogous to the radiative feedback kernel in the ``real world'').
  % % % The linearity of this definition
  % % % permits
  % % % this framework is amenable to a
  % % % traditional definition
  % % % all sources of diabatic heating
  % % % climate sensitivity.

  %%% Proposed work plan
  I propose testing the above
  constraint
  % emergent constraint
  ``candidates'' using
  phases 5 and 6 of the Coupled Model Intercomparison Project
  (CMIP5 and CMIP6).
  I will first compile climatologies
  % of extratropical isentropic slope and thermal diffusivity
  of the CMIP
  ``pre-industrial control'' simulations,
  computing hemisphere-wide metrics of
  isentropic slope and thermal diffusivity.
  % after experimenting with different averaging conventions.
  % computing scalar measures representing hemisphere-average of
  % hemisphere-wide metrics
  % experimenting with different averaging conventions for
  % computing the constraint parameters.
  % % determining constraint parameters.
  I will then compare these metrics against estimates of the ECS
  obtained by regressing the atmospheric energy imbalance against
  1) surface temperature and 2) column-average temperature
  on both a global and per-hemisphere basis \citep{gregory_new_2004}.
  If robust statistical relationships are identified,
  I will compute the same metrics from ``observational'' data
  using the ERA-5 and MERRA-2 reanalysis products
  \citep{gelaro_modern-era_2017,hersbach_era5_2020}
  and use these metrics to establish restricted
  error bounds on ECS for the CMIP5 and CMIP6 ensembles.
  % % ``observations'' metrics in the
  % % to establish error bounds on ECS
  % % implied by the metrics.
  % % % between the constraint metrics
  % % % on both A) a per-hemisphere basis and B) a global basis
  % % I will first
  % % dynamical core ``candidates''
  % % A notable advantage
  % An advantage here is that the constraint
  % is simpler than constraints that rely on
  % internal variability.
  % does not require
  % a variability analysis -- it can
  % be computed with time-averaged
  % as long as the climatological record allows.
  % % temporal analysis of spatial data.
  % % A further advantage is that
  % % the constraint does not require
  % However,
  % \citep[e.g.,][]{bretherton_combining_2020}
  % % temperature climatologies amongst
  % Further, unlike emergent constraints derived from internal variability
  % (e.g.,)
  % However,
  % for emergent constraints
  % Further, ``real-world'' feedbacks are expected to change
  % as the climate changes, as a result of factors beyond
  % pattern effects.
  % % uncertainty in
  % % model-derived estimates of climate sensitivity.
  % % % are useful for the future.
  % % % based around sound physical reasoning.


  % Broader impact
  This work has the potential to reduce uncertainty in model-derived
  estimates of equilibrium climate sensitivity.
  It would also
  % This work would
  test the utility of the dynamical core framework
  for instructing us on real-world relationships
  between the circulation and climate sensitivity.
  % In doing so, it
  % and represents a test of the
  % the sensitivity of the real-world climate.
  % provided by the dynamical core model
  % climate feedbacks and climate swensitivity

  
  \newpage
  ~\vspace{3em}
  \bibliography{refs.bib}

\end{document}
